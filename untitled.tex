O \textit{SoilGrids1km} é um conjunto de mapas de diversas propriedades do solo em diversas profundidades, assim como da classificação taxonômica nos dois sistemas taxonômicos internacionais.

Aproveito para compartilhar uma informação obtida hoje cedo com o colega Tom Hengl do ISRIC: Os mapas do SoilGrids250m ainda não estão disponíveis devido à necessidade da sua atualização, haja vistas que algumas instituições russas acabaram de disponibilizar dados de perfis de solo. Assim, os modelos preditivos serão calibrados novamente, com previsão para os novos mapas para todo o mundo estarem prontos em janeiro próximo. Essa informação é muito importante para pesquisadores como nossa colega Natalia Crespo Mendes, cujo projeto de doutorado envolve o estabelecimento de uma relação da ocorrência de espécies de plantas com o pH do solo em todo o Brasil.

A equipe do ISRIC reconhece que os mapas do SoilGrids250m não são os mais apropriados para o nível de planejamento local. Daí a importância do trabalho das instituições brasileiras. A sugestão da equipe do ISRIC é que os mapas do SoilGrids250m sejam usados pelas instituições brasileiras como variáveis explicativas (covariáveis) em seus modelos de mapeamento do solo. A vantagem disso seria a construção de modelos de mapeamento do solo mais simples, que não requerem o armazenamento de grande quantidade de dados de variáveis explicativas para todo o território nacional. Um exemplo recente desta prática são os mapas produzidos para todo o continente africano, cuja descrição estão no artigo disponível em http://journals.plos.org/plosone/article?id=10.1371/journal.pone.0125814#pone.0125814.e004.

Eu acredito que seja bastante provável que os mapas do SoilGrids250m expliquem a maior parte da variação espacial dos dados de solo no território brasileiro. Caberia a nós encontrar a melhor estratégia para aumentar a acurácia de nossos mapas. Uma delas seria o uso de variáveis preditoras com resolução espacial mais fina (5-30 metros, por exemplo). Contudo, esta estratégia parece pouco eficiente, como nós mostramos em artigo disponível em http://www.sciencedirect.com/science/article/pii/S001670611400456X. Em geral, a maneira mais eficiente de produzir mapas com acurácia melhor do que aquela do SoilGrids250 é a coleta de mais dados de solo, sobretudo na regiões onde possuímos pequena densidade amostral. Uma maneira eficiente de definir os locais onde novos perfis de solo precisam ser descritos, tanto do ponto de vista pedológico como estatístico e financeiro, é a combinação do conhecimento de nossos experientes pedólogos com técnicas computacionais, como aquela que desenvolvemos durante meu projeto de doutorado e está descrita em http://meetingorganizer.copernicus.org/EGU2015/EGU2015-7780.pdf.