O \textit{SoilGrids250m} é um produto do \href{http://www.isric.org/}{ISRIC} -- 
World Soil Information composto por mapas de diversas propriedades do solo em 
diversas profundidades, assim como da classificação taxonômica nos dois sistemas
taxonômicos internacionais, com resolução espacial de 250 metros. A primeira 
versão do \textit{SoilGrids250m} deverá estar disponível gratuitamente em 
janeiro de 2016.

A equipe do ISRIC reconhece que os mapas do \textit{SoilGrids250m} possuem 
inúmeras limitações e, sobretudo, que não são os mais apropriados para a tomada 
de decisão em nível regional/nacional -- o foco do \textit{SoilGrids250m} é o 
nível continental/global. Daí a importância do trabalho das instituições 
brasileiras para produzirem mapas das diversas propriedades do solo para o 
território nacional. Qual seria a utilidade do \textit{SoilGrids250m} nesse 
caso?

A sugestão da equipe do ISRIC é que os mapas do \textit{SoilGrids250m} sejam 
usados como variáveis preditoras (variáveis explicativas, covariáveis) nos 
modelos de mapeamento do solo desenvolvidos em nível regional/nacional. Usar os 
mapas do \textit{SoilGrids250m} como variáveis preditoras nos permitiria 
construir modelos de mapeamento do solo mais simples, que não requerem o 
armazenamento de grande quantidade de dados de variáveis preditoras (imagens de 
satélite, atributos de terreno, entre outras) para todo o território nacional, 
o que representaria um enorme ganho em eficiência operacional. Um exemplo 
recente dessa prática são os mapas produzidos para todo o continente africano 
\cite{HenglEtAl2015}.

Eu acredito que os mapas do \textit{SoilGrids250m} devem explicar a maior parte 
da variação espacial dos dados de solo no território brasileiro. Afinal de 
contas, os mapas do \textit{SoilGrids250m} são resultado do uso dos dados de 
cerca de 5.000 perfis de solo no território brasileiro. Assim, podemos supor 
que o mapa de carbono do \textit{SoilGrids250m} explique ao redor de 50 \% da 
variação espacial do conteúdo de carbono nos primeiros 5 cm do solo no 
território brasileiro. Caberia a nós encontrar a melhor estratégia para produzir
mapas que expliquem mais do que 50 \% da variação espacial -- no caso do carbono
a meta deveria estar entre 70 e 80 \%.

Uma das estratégias comumente empregadas para produzir mapas mais acurados 
consiste em usar variáveis preditoras com resolução espacial mais fina (30 m, 
por exemplo). Contudo, esta estratégia parece pouco eficiente, como nós 
mostramos em artigo publicado na revista
 \href{http://www.sciencedirect.com/science/article/pii/S001670611400456X}{Geoderma}.
O uso de variáveis preditoras com resolução espacial mais fina requer maior 
capacidade de armazenamento e processamento de dados, e dificilmente se traduz
em mapas mais acurados se o volume de dados de solo é pequeno. Assim, a maneira
mais eficiente de produzir mapas com acurácia maior do que aquela do 
\textit{SoilGrids250m} é a coleta de mais dados de perfis de solo, sobretudo na 
regiões onde possuímos pequena densidade amostral.

Em muitos casos, tais dados 
existem mas não estão disponíveis para uso, seja por um ou outro motivo. 


 melhor do que  Uma maneira eficiente de definir os locais onde novos perfis de solo precisam ser descritos, tanto do ponto de vista pedológico como estatístico e financeiro, é a combinação do conhecimento de nossos experientes pedólogos com técnicas computacionais, como aquela que desenvolvemos durante meu projeto de doutorado e está descrita em http://meetingorganizer.copernicus.org/EGU2015/EGU2015-7780.pdf.
